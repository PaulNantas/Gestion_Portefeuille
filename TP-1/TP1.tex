% Options for packages loaded elsewhere
\PassOptionsToPackage{unicode}{hyperref}
\PassOptionsToPackage{hyphens}{url}
%
\documentclass[
]{article}
\usepackage{amsmath,amssymb}
\usepackage{lmodern}
\usepackage{ifxetex,ifluatex}
\ifnum 0\ifxetex 1\fi\ifluatex 1\fi=0 % if pdftex
  \usepackage[T1]{fontenc}
  \usepackage[utf8]{inputenc}
  \usepackage{textcomp} % provide euro and other symbols
\else % if luatex or xetex
  \usepackage{unicode-math}
  \defaultfontfeatures{Scale=MatchLowercase}
  \defaultfontfeatures[\rmfamily]{Ligatures=TeX,Scale=1}
\fi
% Use upquote if available, for straight quotes in verbatim environments
\IfFileExists{upquote.sty}{\usepackage{upquote}}{}
\IfFileExists{microtype.sty}{% use microtype if available
  \usepackage[]{microtype}
  \UseMicrotypeSet[protrusion]{basicmath} % disable protrusion for tt fonts
}{}
\makeatletter
\@ifundefined{KOMAClassName}{% if non-KOMA class
  \IfFileExists{parskip.sty}{%
    \usepackage{parskip}
  }{% else
    \setlength{\parindent}{0pt}
    \setlength{\parskip}{6pt plus 2pt minus 1pt}}
}{% if KOMA class
  \KOMAoptions{parskip=half}}
\makeatother
\usepackage{xcolor}
\IfFileExists{xurl.sty}{\usepackage{xurl}}{} % add URL line breaks if available
\IfFileExists{bookmark.sty}{\usepackage{bookmark}}{\usepackage{hyperref}}
\hypersetup{
  pdftitle={Gestion de Portefeuille},
  pdfauthor={UNG Théophile; POUPARD Paul; NANTAS Paul; SPRIET Thibault},
  hidelinks,
  pdfcreator={LaTeX via pandoc}}
\urlstyle{same} % disable monospaced font for URLs
\usepackage[margin=1in]{geometry}
\usepackage{color}
\usepackage{fancyvrb}
\newcommand{\VerbBar}{|}
\newcommand{\VERB}{\Verb[commandchars=\\\{\}]}
\DefineVerbatimEnvironment{Highlighting}{Verbatim}{commandchars=\\\{\}}
% Add ',fontsize=\small' for more characters per line
\usepackage{framed}
\definecolor{shadecolor}{RGB}{248,248,248}
\newenvironment{Shaded}{\begin{snugshade}}{\end{snugshade}}
\newcommand{\AlertTok}[1]{\textcolor[rgb]{0.94,0.16,0.16}{#1}}
\newcommand{\AnnotationTok}[1]{\textcolor[rgb]{0.56,0.35,0.01}{\textbf{\textit{#1}}}}
\newcommand{\AttributeTok}[1]{\textcolor[rgb]{0.77,0.63,0.00}{#1}}
\newcommand{\BaseNTok}[1]{\textcolor[rgb]{0.00,0.00,0.81}{#1}}
\newcommand{\BuiltInTok}[1]{#1}
\newcommand{\CharTok}[1]{\textcolor[rgb]{0.31,0.60,0.02}{#1}}
\newcommand{\CommentTok}[1]{\textcolor[rgb]{0.56,0.35,0.01}{\textit{#1}}}
\newcommand{\CommentVarTok}[1]{\textcolor[rgb]{0.56,0.35,0.01}{\textbf{\textit{#1}}}}
\newcommand{\ConstantTok}[1]{\textcolor[rgb]{0.00,0.00,0.00}{#1}}
\newcommand{\ControlFlowTok}[1]{\textcolor[rgb]{0.13,0.29,0.53}{\textbf{#1}}}
\newcommand{\DataTypeTok}[1]{\textcolor[rgb]{0.13,0.29,0.53}{#1}}
\newcommand{\DecValTok}[1]{\textcolor[rgb]{0.00,0.00,0.81}{#1}}
\newcommand{\DocumentationTok}[1]{\textcolor[rgb]{0.56,0.35,0.01}{\textbf{\textit{#1}}}}
\newcommand{\ErrorTok}[1]{\textcolor[rgb]{0.64,0.00,0.00}{\textbf{#1}}}
\newcommand{\ExtensionTok}[1]{#1}
\newcommand{\FloatTok}[1]{\textcolor[rgb]{0.00,0.00,0.81}{#1}}
\newcommand{\FunctionTok}[1]{\textcolor[rgb]{0.00,0.00,0.00}{#1}}
\newcommand{\ImportTok}[1]{#1}
\newcommand{\InformationTok}[1]{\textcolor[rgb]{0.56,0.35,0.01}{\textbf{\textit{#1}}}}
\newcommand{\KeywordTok}[1]{\textcolor[rgb]{0.13,0.29,0.53}{\textbf{#1}}}
\newcommand{\NormalTok}[1]{#1}
\newcommand{\OperatorTok}[1]{\textcolor[rgb]{0.81,0.36,0.00}{\textbf{#1}}}
\newcommand{\OtherTok}[1]{\textcolor[rgb]{0.56,0.35,0.01}{#1}}
\newcommand{\PreprocessorTok}[1]{\textcolor[rgb]{0.56,0.35,0.01}{\textit{#1}}}
\newcommand{\RegionMarkerTok}[1]{#1}
\newcommand{\SpecialCharTok}[1]{\textcolor[rgb]{0.00,0.00,0.00}{#1}}
\newcommand{\SpecialStringTok}[1]{\textcolor[rgb]{0.31,0.60,0.02}{#1}}
\newcommand{\StringTok}[1]{\textcolor[rgb]{0.31,0.60,0.02}{#1}}
\newcommand{\VariableTok}[1]{\textcolor[rgb]{0.00,0.00,0.00}{#1}}
\newcommand{\VerbatimStringTok}[1]{\textcolor[rgb]{0.31,0.60,0.02}{#1}}
\newcommand{\WarningTok}[1]{\textcolor[rgb]{0.56,0.35,0.01}{\textbf{\textit{#1}}}}
\usepackage{graphicx}
\makeatletter
\def\maxwidth{\ifdim\Gin@nat@width>\linewidth\linewidth\else\Gin@nat@width\fi}
\def\maxheight{\ifdim\Gin@nat@height>\textheight\textheight\else\Gin@nat@height\fi}
\makeatother
% Scale images if necessary, so that they will not overflow the page
% margins by default, and it is still possible to overwrite the defaults
% using explicit options in \includegraphics[width, height, ...]{}
\setkeys{Gin}{width=\maxwidth,height=\maxheight,keepaspectratio}
% Set default figure placement to htbp
\makeatletter
\def\fps@figure{htbp}
\makeatother
\setlength{\emergencystretch}{3em} % prevent overfull lines
\providecommand{\tightlist}{%
  \setlength{\itemsep}{0pt}\setlength{\parskip}{0pt}}
\setcounter{secnumdepth}{-\maxdimen} % remove section numbering
\usepackage[utf8]{inputenc}
\usepackage{amsmath}
\usepackage{amsfonts}
\usepackage{amssymb}
\ifluatex
  \usepackage{selnolig}  % disable illegal ligatures
\fi

\title{Gestion de Portefeuille}
\usepackage{etoolbox}
\makeatletter
\providecommand{\subtitle}[1]{% add subtitle to \maketitle
  \apptocmd{\@title}{\par {\large #1 \par}}{}{}
}
\makeatother
\subtitle{TP-1: Analyse du CAC40}
\author{UNG Théophile \and POUPARD Paul \and NANTAS Paul \and SPRIET
Thibault}
\date{Février-Mars 2021}

\begin{document}
\maketitle

\hypertarget{les-donnuxe9es}{%
\subsection{Les données}\label{les-donnuxe9es}}

On charge les séries de rendements pour l'indice et les composants de
l'indice.

\begin{Shaded}
\begin{Highlighting}[]
\NormalTok{  ts.all }\OtherTok{\textless{}{-}} \FunctionTok{get.all.ts}\NormalTok{(}\StringTok{\textquotesingle{}CAC40\textquotesingle{}}\NormalTok{, }\AttributeTok{tickers=}\ConstantTok{NULL}\NormalTok{, }\AttributeTok{returns =} \ConstantTok{TRUE}\NormalTok{,}
    \AttributeTok{dt.start =} \FunctionTok{dmy}\NormalTok{(}\StringTok{\textquotesingle{}01Jul2007\textquotesingle{}}\NormalTok{), }\AttributeTok{combine =}\NormalTok{ T)}
  
  \CommentTok{\# bad data for Valeo}
\NormalTok{  ts.all }\OtherTok{\textless{}{-}}\NormalTok{ ts.all[,}\SpecialCharTok{{-}}\DecValTok{17}\NormalTok{]}
  
  \CommentTok{\# keep good data window}
\NormalTok{  ts.all }\OtherTok{\textless{}{-}} \FunctionTok{window}\NormalTok{(ts.all, }\FunctionTok{dmy}\NormalTok{(}\StringTok{\textquotesingle{}01Jul2007\textquotesingle{}}\NormalTok{), }
                   \FunctionTok{dmy}\NormalTok{(}\StringTok{\textquotesingle{}01Jan2009\textquotesingle{}}\NormalTok{))}
  
  \CommentTok{\# merge with cac40 index}
\NormalTok{  cac.index }\OtherTok{\textless{}{-}} \FunctionTok{get.ts}\NormalTok{(}\StringTok{\textquotesingle{}fchi\textquotesingle{}}\NormalTok{, }\StringTok{\textquotesingle{}CAC40\textquotesingle{}}\NormalTok{)}

\NormalTok{  cac.ret }\OtherTok{\textless{}{-}} \FunctionTok{returns}\NormalTok{(cac.index)}
  \FunctionTok{names}\NormalTok{(cac.ret) }\OtherTok{\textless{}{-}} \StringTok{\textquotesingle{}CAC40\textquotesingle{}}
\NormalTok{  ts.all }\OtherTok{\textless{}{-}} \FunctionTok{removeNA}\NormalTok{(}\FunctionTok{cbind}\NormalTok{(ts.all, cac.ret))}
\end{Highlighting}
\end{Shaded}

\begin{Shaded}
\begin{Highlighting}[]
\FunctionTok{plot}\NormalTok{(ts.all[, }\FunctionTok{c}\NormalTok{(}\DecValTok{1}\NormalTok{,}\DecValTok{2}\NormalTok{,}\DecValTok{3}\NormalTok{)], }\AttributeTok{main=}\StringTok{\textquotesingle{}Rendement quotidien\textquotesingle{}}\NormalTok{)}
\end{Highlighting}
\end{Shaded}

\includegraphics{TP1_files/figure-latex/plot-cac-1-1.pdf}

Puis on filtre les points suspects: rendements supérieur à 8 s.d.

\begin{Shaded}
\begin{Highlighting}[]
  \CommentTok{\# flag bad data points: \textgreater{} * \textbackslash{}sigma}
\NormalTok{  good.limit }\OtherTok{\textless{}{-}} \DecValTok{8}\SpecialCharTok{*}\FunctionTok{apply}\NormalTok{(ts.all, }\DecValTok{2}\NormalTok{, sd)}
  
\NormalTok{  ts.bad }\OtherTok{\textless{}{-}}\NormalTok{ ts.all}\SpecialCharTok{*}\ConstantTok{FALSE}
  \ControlFlowTok{for}\NormalTok{(j }\ControlFlowTok{in} \FunctionTok{seq}\NormalTok{(}\FunctionTok{ncol}\NormalTok{(ts.bad))) \{}
\NormalTok{    ts.bad[,j] }\OtherTok{\textless{}{-}} \FunctionTok{abs}\NormalTok{(ts.all[,j]) }\SpecialCharTok{\textgreater{}}\NormalTok{ good.limit[j]}
\NormalTok{  \}}
\NormalTok{  good.index }\OtherTok{\textless{}{-}} \SpecialCharTok{!}\FunctionTok{apply}\NormalTok{(ts.bad,}\DecValTok{1}\NormalTok{,any)}
\NormalTok{  ts.all }\OtherTok{\textless{}{-}}\NormalTok{ ts.all[good.index,]}
\end{Highlighting}
\end{Shaded}

Finalement, on calcule les rendements hebdomadaires:

\begin{Shaded}
\begin{Highlighting}[]
  \CommentTok{\# aggregate returns by week}
\NormalTok{  by }\OtherTok{\textless{}{-}} \FunctionTok{timeSequence}\NormalTok{(}\AttributeTok{from=}\FunctionTok{start}\NormalTok{(ts.all), }
                     \AttributeTok{to=}\FunctionTok{end}\NormalTok{(ts.all), }\AttributeTok{by=}\StringTok{\textquotesingle{}week\textquotesingle{}}\NormalTok{)}
\NormalTok{  ts.all.weekly }\OtherTok{\textless{}{-}} \FunctionTok{aggregate}\NormalTok{(ts.all, by, sum)}

\NormalTok{  ts.stocks }\OtherTok{\textless{}{-}}\NormalTok{ ts.all.weekly[,}\SpecialCharTok{{-}}\DecValTok{40}\NormalTok{]}
\NormalTok{  ts.index }\OtherTok{\textless{}{-}}\NormalTok{ ts.all.weekly[,}\DecValTok{40}\NormalTok{]}
\end{Highlighting}
\end{Shaded}

\includegraphics{TP1_files/figure-latex/plot-cac-2-1.pdf}

\hypertarget{calcul-de-correlation}{%
\subsection{Calcul de correlation}\label{calcul-de-correlation}}

\begin{itemize}
\tightlist
\item
  Calculer la matrice de corrélation des actions de l'indice.
\end{itemize}

\begin{Shaded}
\begin{Highlighting}[]
\NormalTok{cor.stocks }\OtherTok{\textless{}{-}} \FunctionTok{cor}\NormalTok{(ts.all[,}\SpecialCharTok{{-}}\DecValTok{40}\NormalTok{]) }\CommentTok{\# daily}
\NormalTok{cor.hebd.stocks }\OtherTok{\textless{}{-}} \FunctionTok{cor}\NormalTok{(ts.stocks) }\CommentTok{\# weekly}
\end{Highlighting}
\end{Shaded}

\includegraphics{TP1_files/figure-latex/corr-hebd-1.pdf}
\includegraphics{TP1_files/figure-latex/corr-hebd-2.pdf}

\begin{itemize}
\tightlist
\item
  Rechercher des actions fortement corrélées et d'autres qui semblent
  indépendantes. Justifier ces observations en considérant la nature des
  entreprises.
\end{itemize}

On remarque une corrélation plus importante et donc une identification
de clusters plus nette sur la matrice de corrélation des rendements
hebdomadaires. Cela s'explique du fait qu'il peut y avoir des variations
importantes sur une journée (versement des dividendes, annonce de presse
\ldots) mais qu'elles sont lissées sur toute la semaine.\\
Suite à cette remarque, nous décidons de travailler sur la matrice de
corrélation des rendements hebdomadaires.\\
Tout d'abord, la grande majorité des titres sont corrélés positivement,
ce qui met en évience la difficulté de diversifier son portefeuille
d'actifs au sein d'un indice. Il est tout de même intéressant de noter
que le titre \textbf{ei} (Essilor) présente une légère corrélation
négative avec plusieurs autres indices comme : \textbf{bnp},
\textbf{gle} (Société générale) ou encore \textbf{ug} (Peugeot).

\hypertarget{actions-fortement-corruxe9luxe9es}{%
\subsection{Actions fortement corrélées
:}\label{actions-fortement-corruxe9luxe9es}}

\begin{itemize}
\tightlist
\item
  sgo (Saint Gobain) et lg (Lafarge) : toutes les deux dans les
  \textbf{matériaux}
\item
  rno (Renault), ug (Peugeot), ml (Michelin) : milieu de
  \textbf{l'automobile}.
\end{itemize}

\hypertarget{actions-truxe8s-peu-corruxe9luxe9es}{%
\subsection{Actions très peu corrélées
:}\label{actions-truxe8s-peu-corruxe9luxe9es}}

\begin{itemize}
\item
  ora (Orange), et aca (Crédit Agricole) : Télécommunication et Banque
\item
  ora et ml (Michelin) : Télécom et automobile
\item
  tec (Technip) et san (Sanofi Aventis) : Pharmaceutique et
  Ingénierie/Construction
\item
  Choisir 3 titres, et reproduire la figure 3.5, page 35 du manuel de B.
  Pfaff. Commenter les résultats obtenus.
\end{itemize}

\begin{Shaded}
\begin{Highlighting}[]
\NormalTok{roll.edf }\OtherTok{\textless{}{-}} \FunctionTok{timeSeries}\NormalTok{(}\FunctionTok{roll\_cor}\NormalTok{(ts.all}\SpecialCharTok{$}\NormalTok{CAC40,ts.all}\SpecialCharTok{$}\NormalTok{edf,}\DecValTok{10}\NormalTok{),}\AttributeTok{unit=}\StringTok{"CAC \& EDF"}\NormalTok{)}
\FunctionTok{time}\NormalTok{(roll.edf) }\OtherTok{\textless{}{-}} \FunctionTok{time}\NormalTok{(ts.all)}

\NormalTok{roll.tec }\OtherTok{\textless{}{-}} \FunctionTok{timeSeries}\NormalTok{(}\FunctionTok{roll\_cor}\NormalTok{(ts.all}\SpecialCharTok{$}\NormalTok{CAC40,ts.all}\SpecialCharTok{$}\NormalTok{tec,}\DecValTok{10}\NormalTok{),}\AttributeTok{unit=}\StringTok{"CAC \& TEC"}\NormalTok{)}
\FunctionTok{time}\NormalTok{(roll.tec) }\OtherTok{\textless{}{-}} \FunctionTok{time}\NormalTok{(ts.all)}

\NormalTok{roll.bnp }\OtherTok{\textless{}{-}} \FunctionTok{timeSeries}\NormalTok{(}\FunctionTok{roll\_cor}\NormalTok{(ts.all}\SpecialCharTok{$}\NormalTok{CAC40,ts.all}\SpecialCharTok{$}\NormalTok{bnp,}\DecValTok{10}\NormalTok{),}\AttributeTok{unit=}\StringTok{"CAC \& BNP"}\NormalTok{)}
\FunctionTok{time}\NormalTok{(roll.bnp) }\OtherTok{\textless{}{-}} \FunctionTok{time}\NormalTok{(ts.all)}
\end{Highlighting}
\end{Shaded}

\includegraphics{TP1_files/figure-latex/roll-corr-1.pdf}

Etudier la corrélation entre deux indices sur une période n'est pas
suffisant. En effet la corrélation est une valeur dynamique et elle peut
prendre des valeurs très différentes au cours d'une certaine période.
C'est pourquoi la ``rolling correlation'' est une mesure très
intéressante. Cell-ci permet de visualiser l'évolution de la corrélation
entre deux séries temporelles.\\
Dans notre cas, nous avons choisi de comparer les paires suivantes :
\emph{(edf/cac40)}, \emph{(tec/cac40)}, \emph{(bnp.cac40)}.\\
Ces graphiques viennent nuancer nos interprétation de la matrice de
corrélation. En effet on remarque qu'il existe des périodes pendant
lesquelles, un titre va avoir une corrélation positive avec le CAC40
tandis que qu'un autre une corrélation négative.

\hypertarget{analyse-en-composantes-principales}{%
\section{Analyse en composantes
principales}\label{analyse-en-composantes-principales}}

\begin{itemize}
\tightlist
\item
  Effectuer une ACP de la matrice de covariance des rendements
  hebdomadaires
\end{itemize}

\begin{Shaded}
\begin{Highlighting}[]
\NormalTok{cor.returns.hebd }\OtherTok{\textless{}{-}} \FunctionTok{cor}\NormalTok{(ts.stocks)}
\NormalTok{pca.returns.hebd }\OtherTok{\textless{}{-}} \FunctionTok{prcomp}\NormalTok{(cor.returns.hebd)}

\CommentTok{\# normalized eigenvalues}
\NormalTok{norm.ev }\OtherTok{\textless{}{-}}\NormalTok{ pca.returns.hebd}\SpecialCharTok{$}\NormalTok{sdev}\SpecialCharTok{\^{}}\DecValTok{2}
\NormalTok{norm.ev }\OtherTok{\textless{}{-}}\NormalTok{ norm.ev}\SpecialCharTok{/}\FunctionTok{sum}\NormalTok{(norm.ev)}
\NormalTok{large.ev}\FloatTok{.1} \OtherTok{\textless{}{-}}\NormalTok{ norm.ev[}\DecValTok{1}\SpecialCharTok{:}\DecValTok{6}\NormalTok{]}
\FunctionTok{names}\NormalTok{(large.ev}\FloatTok{.1}\NormalTok{) }\OtherTok{\textless{}{-}} \FunctionTok{paste}\NormalTok{(}\StringTok{"PC"}\NormalTok{, }\FunctionTok{seq\_along}\NormalTok{(large.ev}\FloatTok{.1}\NormalTok{))}
\end{Highlighting}
\end{Shaded}

\includegraphics{TP1_files/figure-latex/pca-1.pdf}

\begin{itemize}
\tightlist
\item
  Observer les projections des variables sur les deux premiers vecteurs
  propres, et tenter de fournir une interprétation économique de ces
  facteurs.
\end{itemize}

\begin{Shaded}
\begin{Highlighting}[]
\NormalTok{pca.firstcomp}\FloatTok{.2} \OtherTok{=}\NormalTok{ pca.returns.hebd}\SpecialCharTok{$}\NormalTok{rotation[,}\FunctionTok{c}\NormalTok{(}\DecValTok{1}\NormalTok{,}\DecValTok{2}\NormalTok{)]}
\NormalTok{cor.projected }\OtherTok{=} \FunctionTok{t}\NormalTok{(}\FunctionTok{t}\NormalTok{(pca.firstcomp}\FloatTok{.2}\NormalTok{) }\SpecialCharTok{\%*\%}\NormalTok{ cor.returns.hebd)}
\end{Highlighting}
\end{Shaded}

\includegraphics{TP1_files/figure-latex/2pc-1.pdf}

Dans cette partie, nous allons tenter de montrer que le premier
composant correspond à l'indice du marché, dans notre cas le CAC40. Pour
cela, nous attribuons un poid à chaque titre (entre 0,1) qui correspond
à l'importance de sa projection sur le premier composant. Ensuite nous
calculons pour chaque date la somme des rendements pondérée. Nous
espérons que cette somme soit égale au rendement du cac40.

\begin{Shaded}
\begin{Highlighting}[]
\NormalTok{weights.stocks }\OtherTok{\textless{}{-}} \FunctionTok{unlist}\NormalTok{(}\FunctionTok{lapply}\NormalTok{(cor.projected[,}\DecValTok{1}\NormalTok{],}\ControlFlowTok{function}\NormalTok{(value) }\FunctionTok{abs}\NormalTok{(value)}\SpecialCharTok{/}\FunctionTok{abs}\NormalTok{(}\FunctionTok{sum}\NormalTok{(cor.projected[,}\DecValTok{1}\NormalTok{]))))}
\NormalTok{stocks.returns.weighted }\OtherTok{\textless{}{-}} \FunctionTok{do.call}\NormalTok{(cbind,}\FunctionTok{lapply}\NormalTok{(}\FunctionTok{colnames}\NormalTok{(ts.stocks), }\ControlFlowTok{function}\NormalTok{(stock) ts.stocks[,stock]}\SpecialCharTok{*}\NormalTok{weights.stocks[stock]))}
\NormalTok{ts.index.replicated }\OtherTok{\textless{}{-}} \FunctionTok{colSums}\NormalTok{(}\FunctionTok{t}\NormalTok{(stocks.returns.weighted))}
\end{Highlighting}
\end{Shaded}

\includegraphics{TP1_files/figure-latex/cac40-replicated-1.pdf}

\end{document}
